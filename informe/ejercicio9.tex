\section{Muestra contaminada}
Nos piden analizar aproximar sesgo, varianza y error cuadrático medio para los estimadores de una muestra aleatoria con $b = 1$ y $n = 15$ donde, de manera independiente, un elemento tiene probabilidad $0,005$ de ser \textit{contaminado} (multiplicado por $100$).

\vskip 8pt

Esto nos dice que tenemos, para cada valor $x_{i}$, una variable aleatoria $b_{i}$ de distribución Bernoulli con parámetro $p = 0,005$. Como tenemos $n$ valores en nuestra muestra, tenemos $n$ Bernoullis. Luego, definimos la cantidad de valores contaminados en $X_{n}$ como $Y = \sum_{i=1}^{n}b_{i} \sim Bi(n, p = 0,005)$ porque es una sumatoria de variables aleatorias de Bernoulli, y sabemos su función de densidad:

\begin{center}
	$p_Y(k) = \binom{n}{k}p^k(1-p)^{n-p}$
\end{center}

De esta manera, obtenemos la probabilidad de una muestra $X_1, \dots, X_{15}$ esté contaminada:
\begin{align*}
	P(Y \geq 1) &= 1 - P(Y < 1) \\
				&= 1 - P(Y = 0) \\
				&= 0.07243103119
\end{align*}

Para aproximar el sesgo, la varianza y el error cuadrático medio de cada estimador, procederemos (como en el ejercicio 4) a hacer $1000$ repeticiones en las que calculamos los estimadores (momentos, doble mediana y máxima verosimilitud). A continuación mostramos los resultados obtenidos:

\begin{table}[H]
	\centering
	\begin{tabular}{lcccc}
		\textbf{Estimador} 	& \textbf{Promedio} & \textbf{Sesgo} & \textbf{Varianza} & \textbf{ECM} \\
		$\hat{b}_{mv}$		& $4.309$			& $3.309$		 & $206.119$		 & $217.069$	\\
		$\hat{b}_{mom}$ 	& $1.462$ 			& $0.462$ 		 & $3.938$ 			 & $4.152$		\\
		$\hat{b}_{med}$ 	& $1.005$ 			& $0.005$ 		 & $0.060$ 			 & $0.060$
	\end{tabular}
\end{table}

Como ya discutimos en el punto anterior, el estimador de momentos y el estimador de la doble mediana diluirán la gravedad ejercida por los elementos contaminados (que en probabilidad son pocos) con los valores no contaminados (que serán la gran mayoría). Como con $\hat{b}_{med}$, si hay un elemento contaminado en la muestra, será altamente probable que se produzca una estimación de $b$ menor a alguno de los elementos contaminados y por ende resulte incompatible la estimación del parámetro con la muestra. Si esto no es un problema, y al contrario, se desea estimar $b$ con la menor influencia posible por parte de valores atípicos, el estimador de la doble mediana es la mejor opción al ser insesgado.

\vskip 8pt

Por otro lado, el estimador de máxima verosimilitud se adaptará a cualquier aparición de elementos contaminados. Es decir, el parámetro estimado será siempre el más grande en la muestra, por lo tanto cualquiera estimación de $b$ será compatible con los elementos de la muestra. Es decir, el estimador de máxima verosimilitud es altamente sensible a la aparición de elementos atípicos. Si este tipo de comportamiento es el buscado, este estimador es el indicado.

\subsection{Conclusiones}

Para este caso, con qué estimador quedarse depende de varias cosas. ¿Una muestra contaminada sigue siendo importante o debería ser deshechada? Si la respuesta es que es importante: ¿Qué tanto queremos arriesgarnos a producir un estimador incompatible con la muestra? y ¿Cuánto nos preocupa que nuestra estimación sea consistente? En nuestra opinión, un estimador debe representar con precisión el parámetro de una distribución de donde se extrae una muestra, por lo tanto concluimos que el mejor estimador ante la posibilidad de valores atípicos es el \textbf{estimador de la doble mediana}.