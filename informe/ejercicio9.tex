\section{Ejercicio 9}

Nos piden analizar aproximar sesgo, varianza y error cuadrático medio para los estimadores de una muestra aleatoria con b = 1 y n = 15 donde de manera independiente, a cada elemento se lo multiplica por 100 con probablidad 0,005. (Correr la coma dos lugares a la derecha).

Esto nos dice que tenemos entonces, para cada valor, una variable aleatoria Bernoulli con $p = 1$. Al hacerlo para cada valor de nuestra muestra uniforme, podemos decir que tenemos una v.a. \textit{X = numero de valores contaminados en n repeticiones} con distribucion Binomial(n, p). Damos a continaci\'on, su funci\'on de probabilidad puntual:

\begin{center}[H]
	$p_X(k) = \binom{n}{k}p^k(1-p)^{n-p}$
\end{center}

Con esto podemos saber la probabilidad de una muestra est\'e contaminada:

\begin{center}[H]
	$P(1 \leqslant X) = 1 - P(X = 0) = 0.07243103119$
\end{center}

Para aproximar el sesgo, la varianza y el error cuadr\'atico medio de cada estimador, procederemos (como en el ejercicio 4) a hacer 1000 repeticiones en las que calculamos los estimadores (momentos, doble mediana y m\'axima verosimilitud). A continuaci\'on mostramos los resultados obtenidos.

\begin{table}[H]
	\centering
	\begin{tabular}{lccc}
		\textbf{Estimador} & $\hat{\theta}_{mv}$ & $\hat{\theta}_{mom}$ & $\hat{\theta}_{med}$ \\
		Promedio  & $4.309143770088162$ 			& $1.4625167466253193$ 		& $1.0052100231711671$ \\
		Sesgo     & $3.309143770088162$ 			& $0.4625167466253193$ 		& $0.005210023171167144$ \\
		Varianza  & $206.11915535591314$ 			& $3.9386781273955753$ 		& $0.060947650789632726$ \\
		ECM       & $217.06958784702644$ 			& $4.152599868304445$ 		& $0.06097479513107683$
	\end{tabular}
\end{table}

Con estos valores, preferimos el estimador de momentos ya que tiene un el sesgo y el error cuadratico medio mas chico con respecto a diferencia de los demas, en los que vemos que hay un sesgo considerable con respecto al valor real de $b$.