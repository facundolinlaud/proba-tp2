\section{Ejercicio 8}

Se nos pide calcular los Estimadores de la siguiente muestra

\begin{center}
0,917 0,247 0,384 0,530 0,798 0,912 0,096 0,684 0,394 20,1 0,769 0,137 0,352 0,332 0,670
\end{center}

Mostramos a continuaci\'on, cuales son los valores obtenidos:

\begin{table}[h]
	\centering
	\begin{tabular}{lccc}
		\textbf{Tipo de Estimador} & \textbf{Valor} \\
		m\'axima versimilitud  &  $20,1$ \\
		momentos & $3.64293$ \\
		doble mediana & $1.06$
	\end{tabular}
\end{table}

Podemos observar la gran diferencia hay entre los estimadores. El estimador de m\'axima verosimilitud es el mayor de todos, mientras que, tanto el estimador de doble mediana como el estimador de momento son mucho mas chicos. Creemos que esto se debe al valor at\'ipico (20,1) que se encuentra en la muestra, ya que el esimador de ma\'ima verosimilitud toma el m\'aximo de toda la muestra. Le sigue el estimador de momentos (con 3,64293). Este \'ultimo es un poco mas grande ya que nuestros ya que al ser el promedio de los valores, es suceptible a valores at\'ipicos. Finalmente, est\'a la doble mediana, en la que se toma la mediana de la muestra (0,530) y se lo multiplica por 2. 