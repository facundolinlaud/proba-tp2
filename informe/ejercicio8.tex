\section{Análisis de una muestra específica}
Se nos pide calcular los Estimadores de la siguiente muestra

\begin{center}
$0,917, 0,247, 0,384, 0,530, 0,798, 0,912, 0,096, 0,684, 0,394, 20,1, 0,769, 0,137, 0,352, 0,332, 0,670$
\end{center}

Y estos son los estimadores de la muestra, asumiendo que proviene de una distribución uniforme de parámetros $a = 0$ y $b$:

\begin{table}[h]
	\centering
	\begin{tabular}{lccc}
		\textbf{Estimador} 	& \textbf{Valor} 	\\
		$\hat{b}_{mv}$  	& $20,1$ 			\\
		$\hat{b}_{mom}$		& $3.64293$ 		\\
		$\hat{b}_{med}$		& $1.06$
	\end{tabular}
\end{table}

Podemos observar que en la muestra, todos los elementos están entre $0$ y $1$, con la excepción de uno que vale $20,1$. Debido a que $\hat{b}_{mv} = max\{x_1, \dots, x_n\}$, es decir, el estimador de máxima verosimilitud toma la forma del elemento más grande en la muestra, este estimador brindará el valor $20,1$ que será compatible con la forma de los datos. Sin embargo, $\hat{b}_{mom}$ toma en cuenta todos los valores de la muestra para determinar su valor, por lo tanto brinda cierta resistencia ante un valor grande y aislado como $20,1$, tomando un valor inferior a este. Lo mismo sucede con $\hat{b}_{med}$.

\vskip 8pt

El estimador más confiable para este caso es \textbf{el estimador de máxima verosimilitud}.